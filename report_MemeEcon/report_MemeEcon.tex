\documentclass[12pt]{article}
\usepackage{ucs}
\usepackage[utf8x]{inputenc}
\usepackage[T1]{fontenc}
\usepackage{lmodern}
\usepackage[english]{babel}
\usepackage[nottoc]{tocbibind}
\usepackage[left=2.5cm,right=2.5cm,top=2.5cm,bottom=2.5cm,]{geometry}
\usepackage{graphicx}
\usepackage{nameref}
\usepackage[hidelinks]{hyperref}
\usepackage{sidecap}
\usepackage{wrapfig}
\usepackage[bottom,hang]{footmisc}
\usepackage{acronym}
\usepackage{multirow}
\usepackage{color}
\usepackage{capt-of}
\usepackage{array,				%better tables
			tabularx,			%instead of tabular*
			booktabs,			%tables for good publications
}
\usepackage{afterpage,hyperref}
\usepackage{listings}
\lstset{
    basicstyle=\ttfamily\footnotesize,
    numbers=left,
    xleftmargin=15pt,
}

\usepackage{subfigure}
%\usepackage{float}


% TabularX column defintions
\newcolumntype{L}[1]{>{\raggedright\let\newline\\\arraybackslash\hspace{0pt}}m{#1}}
\newcolumntype{C}[1]{>{\centering\let\newline\\\arraybackslash\hspace{0pt}}m{#1}}
\newcolumntype{R}[1]{>{\raggedleft\let\newline\\\arraybackslash\hspace{0pt}}m{#1}}

%\captionsetup[table]{aboveskip=4pt}
%\captionsetup[table]{belowskip=0pt}
%\captionsetup{hypcap=false}



%--------------------------MAKETITLE-------------------------------%
\title{Introduction to Blockchain Technology:\\
    Meme Economy Blockchain}
\date{\today}
\author{Rahul Agrawal, Akshay Katyal, Mehmed Mustafa\\
Anant Sujatanagarjuna, Steffen Tunkel, Chris Warin}

\begin{document}
%--------------------------BEGINNING-------------------------------%
\pagenumbering{roman}
\maketitle
\thispagestyle{empty}
\begin{abstract}% Steffen
Here could go our kick-ass abstract.

\end{abstract}

\newpage
\tableofcontents
\newpage

%-----------------------SECTION START------------------------
\pagenumbering{arabic}


%\setcounter{table}{0}

\section{Introduction} \label{sec:Intro}% Steffen
% Motivation
    % Meme Culture
% Idea / Concept


\section{Foundations} \label{sec:Foundations} %Akshay
% section intro
This section gives information regarding the software and interfaces used during the development of our Blockchain application.

\subsection{RESTful Applications}

A RESTful application is an Application program interface (API) that uses HTTP requests to access the resources. An API for an application is code that allows two software programs to communicate with each other.
A RESTful application can also be said as a web service based on REST, abbreviated as Representational State Transfer. The application breaks down user requests into a small series of modules, where each module addresses the part of the request. The Restful application takes JSON strings as the input given as: \emph{{“name”: “example”}}. As discussed earlier a RESTful application uses HTTP methodologies to access the resources (object, file, or block) given as follows:
\begin{itemize}
    \item GET: Request to retrieve a resource
    \item PUT: Request to update a resource
    \item POST: Request to create a new resource
    \item DELETE: Request to remove a resource
\end{itemize}
\begin{figure}
    \centering
    \includegraphics[width=5cm, height=3.5cm]{../report_MemeEcon/images/restful_api.jpg}
    \caption{RESTful Application(\href{https://b2evolution.net/man/rest-api/}{Source})}
    \label{fig:Restful}
\end{figure}

\subsection{Flask Framework}
Flask is a Python-based web-development framework, also called as Web Server Gateway Interface (WSGI). Used as a simple wrapper around Werkzeug and Jinja and now has become the most popular Python web application framework. The Flask framework is used to start a local development server on the user system, this is done by importing \textbf{flask} run in the application by the use of the FLASK\_APP environment variable.
\begin{flushleft}
The flask uses the \emph{route()} decorator to determine which function will execute in the program code. The flask handles the request and response between the local server nodes.
\end{flushleft} \par
\textbf{Sample Flask example:}
\begin{lstlisting}
@app.route('/example/')
def example():
    return {'hello': 'world'}

python ./example.py
 * Running on http://127.0.0.1:5000/
\end{lstlisting}
\begin{figure}
    \centering
    \includegraphics[width=3.5cm, height=3.5cm]{../report_MemeEcon/images/flask.png}
    \caption{Flask(\href{https://palletsprojects.com/p/flask/}{Source})}
    \label{fig:Flask}
\end{figure}

\subsection{Postman}
Postman is a platform for API development. The Postman interface has a bunch of features that help in the building of an API given as:
\begin{itemize}
    \item API Client: Helps to send requests within postman
    \item Automated Testing: Helps in the testing of the API
    \item Monitor: Helps to check the performance of the API
    \item Documentation: Helps to create documentation for our API
\end{itemize}
\begin{flushleft}
The API Client allows the user to Send Requests and View response generated by the execution of the REST queries within the Postman. It also allows the user to visualize data by the use of Postman Visualizer, in our application the Postman helps to visualize the Memes and the Meme Formats. The platform also allows the user to create a test suite for their application by the use of the Collections feature, where each collection consists of multiple HTTP requests.
\end{flushleft}
\begin{figure}
    \centering
    \includegraphics[width=3.5cm, height=3.5cm]{../report_MemeEcon/images/postman.png}
    \caption{Postman(\href{https://www.postman.com/resources/media-assets/}{Source})}
    \label{fig:Postman}
\end{figure}

\section{Related Work} %Rahul
% Sub-reddit approach

\section{Approach} \label{sec:Approach}
% section intro and system diagram %Chris because he likes to make good looking stuff. And Mehmed too. Anant also has to be involved.
\subsection{Block Design} %Mehmed

Each block inside the chain has the following fields:

\begin{itemize}
  \item index - the index of the block.
  \item minerID - the id of the miner. This identification is used when distributing mining rewards.
  \item previous\_hash - the hash of the previous block inside the chain
  \item proof\_of\_work - a nonce value (magic number) which makes the hash of the block to match the difficulty pattern
  \item timestamp - an instance of time showing the creation time of the block
  \item transaction\_counter - the number of transactions inside the block
  \item transactions - a list which holds all transactions inside the block
\end{itemize}

The genesis block has "0" value on all fields and an empty list for the transactions. The hash value of the genesis block does not match any hash patterns and when validating the chain the genesis block is skipped. It should be noted that even if the block is not skipped, since all values inside the genesis block are predefined, the hash value will always be the same value and could be hard coded.

All consecutive blocks, after the genesis block, have their proper field values. The minerID field is a special field because it's value is different for each node. Each node assigns its ID to the minerID field and tries to find a nonce value with this assigned specific minerID. Also each time a new transaction is added to the memory pool of transactions, the finding nonce value process is restarted, since the transactions list is updated.


\subsection{Peer to Peer Network} %Akshay
In a blockchain peer-to-peer connection, any node connects to any other node in the network collection of nodes. The nodes in the network are connected in the form of mesh network as shown in the fig, where each node is connected to all of its neighboring nodes and each node contains a copy of the complete blockchain. As a new node is connected to a blockchain network then all the nodes in the network are notified about the newly connected node in the network. Moreover, the newly connected node receives a copy of the complete blockchain and also receives the list of pending transactions in the blockchain. \par
\begin{flushleft}

In our application meme economy, the peer to peer concept is demonstrated by the use of the Postman as:
\begin{itemize}
    \item The core node must run on port 5000 in order to initialize the network, the port 5000 is selected specifically to prevent creation of multiple networks at the same time.
    \item Add other nodes on different port addresses.
    \item Now to show the peer to peer concept, we send a POST request using the connect\_to\_node method, if the request is successful we get a response showing Connection Successful on the console with status code 200.
\end{itemize}
\end{flushleft}
As the connection is successful, the newly connected node receives a copy of the blockchain and list transitions and pending transactions in the blockchain network.
\begin{figure}
    \centering
    \includegraphics[width=8cm, height=5cm]{../report_MemeEcon/images/p2p.png}
    \caption{Peer-to-Peer(\href{https://www.researchgate.net/publication/320127088_Blockchain_Implementation_Quality_Challenges_A_Literature_Review/figures?lo=1&utm_source=google&utm_medium=organic}{Source})}
    \label{fig:P2P}
\end{figure}
\subsection{Transactions} %Mehmed

In the Block Design section it was discussed that each block has a list of transactions. There are 5 different kind of possible transactions:
\begin{itemize}
  \item add\_memeFormat - Used to add a new meme format to the network
  \item add\_meme - Used to add a new meme to the network
  \item add\_upvote - Used to upvote a meme inside the network
  \item sell\_ownership - Used to sell the ownership of a meme format
  \item purchase\_ownership - Used to purchase the ownership of a meme format
\end{itemize}

\subsubsection{Raw-JSON body input formats of transactions}

\begin{itemize}
\item add\_memeFormat - \{"imagePath" : "imagePathValue",\\ "name" : "nameValue"\}

\item add\_meme - \{"imagePath" : "imagePathValue",\\ "name" : "nameValue",\\ "memeFormat" : "memeFormatID"\}

\item add\_upvote - \{"imageVoteId":"memeID",\\ "upvoteID" : "upvoteID"\}

\item sell\_ownership - \{"ownershipSaleOfferID" : "ownershipSaleOfferID",\\ "memeFormat" : "memeFormatID",\\ "saleAmount" : "saleAmount"\}

\item purchase\_ownership - \{"ownershipPurchaseID" : "ownershipPurchaseID",\\ "ownershipSaleOfferID" : "ownershipSaleOfferID"\}
\end{itemize}

\subsubsection{Transaction requirements and validation}

This subsection discusses the requirements of each transaction. Once all of the requirements for a specific transaction are met, then the transaction is validated and ready to be mined in the next block.

\begin{itemize}
\item add\_memeFormat - There are not any requirements for this transaction.

\item add\_meme - Each meme to be uploaded must belong to a specific meme format. This meme format must be already mined inside a block and appended to the chain.

\item add\_upvote - The meme to be up voted must be already mined inside a block and appended to the chain. The node which up votes must have enough credits available in its wallet.

\item sell\_ownership - The meme format to be sold must be already mined inside a block and appended to the chain. The node which creates this transaction must have the ownership of the meme format to be sold.

\item purchase\_ownership - The meme format to be purchased must be already offered for a sale, mined and appended to the chain. The purchasing node must have enough credits in order to buy the meme format. The worth of the meme format is specified inside the sell\_ownership transaction.
\end{itemize}


\subsection{Mining and Consensus} %Steffen and Anant
\subsection{Wallet and Rewards} %Anant
\subsection{Visualization} %Chris


\section{Outlook} \label{sec:Outlook} %Chris

\section{Conclusion} \label{sec:Conclusion} %Chris



\section{Template Section} \label{sec:Template}
    This template should give us a first version we can start of with. This last section should support us in writing a more coherent paper together. Therefore I put some guidelines for the writing in section~\ref{sec:Guidelines}. Also there are some example for the use of functionalities in Section~\ref{sec:Examples}. You can copy them and adapt them the way you need them. Just leave this section here for now.


    \subsection{Guidelines} \label{sec:Guidelines}
        \begin{itemize}
        \item \textbf{pushing to the repo} \\
            When you push the newest version to the repo, please leave out the files created by the compiler (besides the pdf). The report on the repo just needs the tex-file, the bib-file and maybe the most current pdf-file. Of course push the changes to the sub folders if you added images or sources.

        \item \textbf{references} \\
            For every section, subsection, figure, or table you include give it a label. Therefore everyone can refer to it later. It can be done by \textit{$\backslash$label\{marker\}}. The marker should declare the type of the object and a short (one word in the best case) name for the object. The types are "sec:" for a section, "fig:" for a figure and "tab:" for a table. You can refer to them then by \textit{$\backslash$ref\{fig:example\}}. Also you should use a $\sim$ symbol before instead of normal space to avoid line breaks there.

        \item \textbf{citations} \\
            For cititions the BibTex code for the source needs to be in the 'literature\_list.bib' file. You can usually get them pretty easily from \textit{Google Scholar}. Please safe all papers/sources you used in the "sources" folder in addition. The citation can then be made by \textit{$\backslash$cite\{antonopoulos2017mastering\}} for example. Please use the $\sim$ here too. The result then looks like this~\cite{antonopoulos2017mastering}.


        \item \textbf{abbreviations} \\
            Please use the $\backslash$ac\{...\} command to handle abbreviations. You can define them at the end of the document. Here is one example... When used the first time it automatically defines the abbreviation: \ac{AI}. For all furhter times it just prints: \ac{AI}. Also the plural is possible \acp{AI}

        \end{itemize}


    \subsection{Examples} \label{sec:Examples}

    \begin{figure}[t]
        \centering
        \subfigure[Meme Format]{\includegraphics[width=.49\textwidth]{../memes/woman-yelling-at-cat/template.jpg}}
        \subfigure[Meme]{\includegraphics[width=.49\textwidth]{../memes/woman-yelling-at-cat/4.jpg}}
        \caption{Example for 2 subfigures.}
        \label{fig:MemeFormat-Meme}
    \end{figure}



    \begin{figure} [t]
        \centering
        \includegraphics[width=.3\textwidth]{../memes/drake/4.jpg}
        \caption{Example for a single figure.}
        \label{fig:drake}
    \end{figure}



    \begin{table}[t]
        \centering
        %\captionof{table}{The tasks for the different prototypes.}
        \begin{tabular}{|C{.07\textwidth}|L{.41\textwidth}|L{.41\textwidth}|}
            \hline & \textbf{SVM} & \textbf{Neural Network} \\
            \hline MR-1 & Permutation of training \& test features & Permutation of input channels (RGB channels) for
training \& test data \\
            \hline MR-2 & Permutation of order of training instances & Permutation of the convolution operation order for
training \& test data \\
            \hline MR-3 & Shifting of training \& test features by a constant (only for RBF kernel) & Normalizing the test data \\
            \hline MR-4 & Linear scaling of the test features (only for linear kernel) & Scaling the test data by a constant \\
            \hline
        \end{tabular}

        \captionof{table}{Content totally out of context, literally just as an example for a table.}
        \label{tab:relations}
    \end{table}

        \begin{center}
    \begin{lstlisting}
    # Computes the hash of the Block
    def compute_hash(self):
        # self.__dict__ -> all variables inside the Block class
        encoded_block = json.dumps(self.__dict__, sort_keys=True).encode()
        return hashlib.sha256(encoded_block).hexdigest()
    \end{lstlisting}
    \end{center}




\newpage

%-------------------References---------------------------%
\section*{Abbreviations and Acronyms}
\addcontentsline{toc}{section}{Abbreviations and Acronyms}
\begin{acronym}[Bash]
\acro{AI} {\textit{Artificial Intelligence}}
\acro{ML} {\textit{Machine Learning}}


\end{acronym}

%----------------List-of-Tables--------------------%
%Comment the following lines out if you dont have tables or figures
\listoftables
\listoffigures
\bibliographystyle{plain}
\bibliography{literature_list}
\end{document}
